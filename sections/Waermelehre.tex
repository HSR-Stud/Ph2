\newpage
\section{Wärmelehre}

\subsection{Temperatureinheiten}				%Temperatureinheiten
%\begin{table}[h!]
	\begin{tabular}{ | m{9cm} | m{9cm}  | }
		\hline
		Formeln & Einheiten \\ \hline
		\hline
		%\begin{minipage}[t]{5cm}
		\begin{itemize}
			\item $^\circ\text{C}=\dfrac{^\circ\text{F}-32}{1.8}$
			\item $^\circ\text{F}=^\circ\text{C}\cdot 1.8+32$
			\item $^\circ\text{C}=K-273.15$
			\item $K=^\circ\text{C}+273.15$
			
		\end{itemize}
		%\end{minipage}
		&
		%\begin{minipage}{5cm}
		\begin{itemize}
			\item $^\circ\text{C}=$ Temperatur in Celsius
			\item $^\circ\text{F}=$ Temperatur in Fahrenheit
			\item $K=$ Temperatur in Celsius
		\end{itemize}
		%\end{minipage}
		\\ \hline
	\end{tabular}
%\end{table}

\subsection{Molare Masse, Molmasse}				%Molare Masse, Molmasse
%\begin{table}[h!]
	\begin{tabular}{ | m{9cm} | m{9cm}  | }
		\hline
		Formeln & Einheiten \\ \hline
		\hline
		%\begin{minipage}[t]{5cm}
		\begin{itemize}
			\item $M=\dfrac{m}{n}=N_{A}\cdot m_M$
			
			
		\end{itemize}
		%\end{minipage}
		&
		%\begin{minipage}{5cm}
		\begin{itemize}
			\item $m=[kg]$
			\item $n=[mol]$
			\item $M=[\frac{kg}{mol}]$
			\item $N_{A}=6.022\cdot 10^{23}$ $[\dfrac{1}{mol}] $
			\item $m_M=[kg]$
			
		\end{itemize}
		%\end{minipage}
		\\ \hline
	\end{tabular}
%\end{table}

\subsection{Längen und Volumenänderung}				%Längen und Volumenänderung
%\begin{table}[h!]
	\begin{tabular}{ | m{6cm} | m{8cm} | m{4cm} | }
		\hline
		Abbildung & Formeln & Einheiten \\ \hline
		\hline
		\begin{minipage}{.3\textwidth}
			\tabImg[width=6.0cm]{images/Deltal}
		\end{minipage}
		&
		%\begin{minipage}[t]{5cm}
		\begin{itemize}
			\item $\Delta l=\alpha\cdot l\cdot \Delta T$	
			\item $\Delta V=\gamma\cdot V\cdot \Delta T$	
			\item $\alpha,\gamma=Ausdehnungskoeffizienten$
			\item $\gamma=3\cdot \alpha$ (Gilt bei isotopen Materialien)
			\item Isotop = In allen Richtungen gleiche Eigenschaften	
		\end{itemize}
		%\end{minipage}
		& 
		%\begin{minipage}{5cm}
		\begin{itemize}
			\item $\Delta l,l= [m]$
			\item $\Delta V,V=[m^3]$
			\item $\alpha,\gamma=[\frac{1}{K}]$	
			\item $\Delta T=[K]$
		\end{itemize}
		%\end{minipage}
		\\ \hline
	\end{tabular}
%\end{table}

\subsection{Thermische Spannung, Hookesches Gesetz}				%Thermische Spannung, Hookesches Gesetz
%\begin{table}[h!]
	\begin{tabular}{ | m{6cm} | m{8cm} | m{4cm} | }
		\hline
		Abbildung & Formeln & Einheiten \\ \hline
		\hline
		\begin{minipage}{.3\textwidth}
			\tabImg[width=6.0cm]{images/Spannung}
		\end{minipage}
		&
		%\begin{minipage}[t]{5cm}
		\begin{itemize}
			\item Ohne Behinderung: $\Delta l=\alpha\cdot l\cdot \Delta T$	
			\item Mit Behinderung: $\Delta l=0$	
			\item $\sigma=E\cdot \dfrac{\Delta l}{l}$
			\item $\sigma=E\cdot \alpha\cdot \Delta T$
			\item E = Elastizitätsmodul
			\item $\sigma=$Thermische Spannung
		\end{itemize}
		%\end{minipage}
		& 
		%\begin{minipage}{5cm}
		\begin{itemize}
			\item $E= [\frac{N}{m^{2}}]=Pa$
			\item $\sigma=[\frac{N}{m^{2}}]=Pa$
			\item $\Delta l,l= [m]$
			\item $\Delta V=[m^3]$
			\item $\alpha=[\frac{1}{K}]$	
			\item $\Delta T=[K]$
		\end{itemize}
		%\end{minipage}
		\\ \hline
	\end{tabular}
%\end{table}
\subsubsection{Thermodynamische Systeme}
	$\Rightarrow$ offen = Austausch von Energie und Austausch von Material (z.B. Wärmetauscher, Kompressor, Gasturbine)\\
	$\Rightarrow$ geschlossen = Austausch von Energie und kein Austausch von Material (z.B. Heizkreislauf, Kühlschrank)\\
	$\Rightarrow$ abgeschlossen = kein Austausch von Energie und kein Austausch von Material (z.B. Ideale Thermosflasche)\\ 	
	$\Rightarrow$ adiabatisch = kein Wärmeaustausch, kein Materialaustausch, aber Energieaustausch (z.B. Kompressor)\\

\newpage

\subsection{Thermische Zustandsgleichung, ideales Gas}				%Thermische Zustandsgleichung
%\begin{table}[h!]
	\begin{tabular}{ | m{9cm} | m{9cm}  | }
		\hline
		Formeln & Einheiten \\ \hline
		\hline
		%\begin{minipage}[t]{5cm}
		\begin{itemize}
			\item $p\cdot V=N\cdot k\cdot T$	
			\item $p\cdot V=n\cdot R\cdot T$
			\item $p\cdot V=\dfrac{m}{M}\cdot R\cdot T$
			\item $R=N_{A}\cdot k$ 
			\item $n=\dfrac{m}{M}$
			\item R = Universelle Gaskonstante
			\item k = Boltzmann-Konstante
			\item $N_{A}$ = Avogadro-Zahl
			\item {\color{red}1 atü = $98066.5$ $ [Pa]$}
			\item {\color{red}1 Torr = $133.322$ $ [Pa] = 1[mm]\cdot g\cdot \rho_{Hg}$}
		\end{itemize}
		%\end{minipage}
		&
		%\begin{minipage}{5cm}
		\begin{itemize}
			\item $p= [\frac{N}{m^{2}}]=Pa$
			\item $V=[m^3]$
			\item $N=[1]=$ Anzahl Moleküle
			\item $n=[1]=$ Anzahl Mole
			\item $M=[\frac{kg}{mol}]=$ Molare Masse
			\item $T=[K]$
			\item $k=1.381\cdot 10^{-23}$ $[\frac{J}{K}]$
			\item $R=8.314$ $[ \frac{J}{mol\cdot K} ]$
			\item $N_{A}=6.022\cdot 10^{23}$ $[\dfrac{1}{mol}] $
			\item $\rho_{Hg}=13595[\frac{kg}{m^3}]$
			\item $g=9.81[\frac{m}{s^2}]$
			
			
			
		\end{itemize}
		%\end{minipage}
		\\ \hline
	\end{tabular}
%\end{table}

	 Ideales Gas $\Rightarrow$  \textbf{1.)} $(p\cdot V=const)$            Teilchen sind Massepunkte \textbf{2b.)} Teilchen üben keine gegenseitigen Kräfte aus\\ 
	 Reales Gas $\Rightarrow$ \textbf{1.)}  Teilchen dehnen sich aus $\rightarrow$ Volumen ist jetzt: (V-b) \textbf{2.)} Teilchen wirken Kräfte aus $(p=p_{0}+\dfrac{a}{V^{2}})$\\
	 
	 \newpage
	

\subsection{Van der Waal'sche Zustandsgleichung, reale Gase}				%Van der Waal'sche Zustandsgleichung, reale Gase
%\begin{table}[h!]
	\begin{tabular}{ | m{6cm} | m{6cm} | m{6cm} | }
		\hline
		Abbildung & Formeln & Einheiten \\ \hline
		\hline
		\begin{minipage}{.3\textwidth}
			\tabImg[width=6.0cm]{images/vanderwaal}
		\end{minipage}
		&
		%\begin{minipage}[t]{5cm}
		\begin{itemize}
			\item(allg.): $p=\dfrac{n\cdot R\cdot T}{V_{m}-b}-n^{2}\cdot \dfrac{a}{V_{m}^{2}}$ 
			\item(1 mol): $p=\dfrac{R\cdot T}{V_{m}-b}-n^{2}\cdot \dfrac{a}{V_{m}^{2}}$
			\item $V=V_{m}\cdot n$
			\item a = Kohäsionsdruck (materialabhängig)
			\item b = Kovolumen (materialabhängig)
		\end{itemize}
		%\end{minipage}
		& 
		%\begin{minipage}{5cm}
		\begin{itemize}
			\item $p= [\frac{N}{m^{2}}]=Pa$
			\item $V=[m^3]$
			\item $n=[1]=$ Anzahl Mole
			\item $V_{m}=[\frac{m^3}{mol}]=$ molares Volumen
			\item $T=[K]$
			\item $a=[\frac{10^{-3}\cdot Pa\cdot m^{6}}{mol^{2}}]$
			\item $b=[\frac{10^{-6}\cdot m^{3}}{mol}]$
			
		\end{itemize}
		%\end{minipage}
		\\ \hline
	\end{tabular}
%\end{table}

\subsection{Mittlere freie Weglänge}				%Mittlere freie Weglänge
%\begin{table}[h!]
	\begin{tabular}{ | m{6cm} | m{7.5cm} | m{4.5cm} | }
		\hline
		Abbildung & Formeln & Einheiten \\ \hline
		\hline
		\begin{minipage}{.3\textwidth}
			\tabImg[width=6.0cm]{images/Weglaenge}
		\end{minipage}
		&
		%\begin{minipage}[t]{5cm}
		\begin{itemize}
			\item\={l} $ =\dfrac{1}{\sqrt{2}\cdot n\cdot \pi\cdot d^{2}}=\dfrac{R\cdot T}{\sqrt{2}\cdot \pi\cdot d^2\cdot p\cdot N_A}$ 
			\item $n=\dfrac{N}{V}=\dfrac{p\cdot N_A}{R\cdot T}$
			
		\end{itemize}
		%\end{minipage}
		& 
		%\begin{minipage}{5cm}
		\begin{itemize}
			\item\={l} $ =[m]$ 
			\item$n=\dfrac{Teilchen}{Volumen}=[\dfrac{1}{m^{3}}]$
			\item$d=[m]$
			\item $N=[1]$
			\item $V=[m^3]$
			\item $p=[Pa]$
			\item $T=[K]$
			\item $N_{A}=6.022\cdot 10^{23}$ $[\dfrac{1}{mol}]$
			\item $R=8.314$ $[ \frac{J}{mol\cdot K} ]$
			
		\end{itemize}
		%\end{minipage}
		\\ \hline
	\end{tabular}
%\end{table}

\newpage

\subsection{kinetische Gastheorie}				%kinetische Gastheorie
%\begin{table}[h!]
	\begin{tabular}{ | m{6cm} | m{6cm} | m{6cm} | }
		\hline
		Abbildung & Formeln & Einheiten \\ \hline
		\hline
		\begin{minipage}{.3\textwidth}
			\tabImg[width=6.0cm]{images/Gastheorie}
		\end{minipage}
		&
		%\begin{minipage}[t]{5cm}
		\begin{itemize}
			\item$p=\dfrac{F}{A}=\dfrac{1}{3}\cdot \dfrac{_{N_{A}\cdot v^{2}\cdot m}}{V}$ 
			\item$E_{kin}=\dfrac{m\cdot v^{2}}{2}=\dfrac{3}{2}\cdot k\cdot T$
			
		\end{itemize}
		%\end{minipage}
		& 
		%\begin{minipage}{5cm}
		\begin{itemize}
			\item $p= [\frac{N}{m^{2}}]=Pa$
			\item $F=[N]$
			\item $A=[m^2]$
			\item $V=[m^3]$
			\item $m=[kg]$
			\item $v=[m\frac{m}{s}]$
			\item $T=[K]$
			\item $k=1.381\cdot 10^{-23}$ $[\frac{J}{K}]$
			\item $N_{A}=6.022\cdot 10^{23}$ $[\dfrac{1}{mol}] $
			
		\end{itemize}
		%\end{minipage}
		\\ \hline
	\end{tabular}
%\end{table}

\subsection{Thermodynamik}				%Thermodynamik
%\begin{table}[h!]
%	\begin{center}
	\begin{tabular}{ | m{10cm} | m{8cm}  | }
		\hline
		Formeln & Einheiten \\ \hline
		\hline
		%\begin{minipage}[t]{5cm}
		\begin{itemize}
			\item Erster Hauptsatz der Thermodynamik: $dU=\delta W+\delta Q$ 	
			\item $\Delta Q=m\cdot c\cdot \Delta T$
			\item $m\cdot c=C$
			\item $\Delta T=T_{2}-T_{1}\Rightarrow$ {\color{red} So, dass es positives $\Delta T$ gibt}
			\item $Q_{ab}=Q_{zu}$
			\item $Q_{s}=$ Schmelzwärme $= q_{s}\cdot m$ 	
			\item $Q_{v}=$ Verdampfungswärme $= q_{v}\cdot m$
			\item $q_{s}=$ spez. Schmelzwärme
			\item $q_{v}=$ spez. Verdampfungswärme 	
			\item $T_{M}=$ Mischtemparatur zweier Stoffe
			\item $T_{M}=\dfrac{m_{1}\cdot c_{1}\cdot T_{1}+m_{2}\cdot c_{2}\cdot T_{2}}{m_{1}\cdot c_{1}+m_{2}\cdot c_{2}}$
		\end{itemize}
		%\end{minipage}
		&
		%\begin{minipage}{5cm}
		\begin{itemize}
			\item $dU=$innere Energie$=[J]$
			\item $\delta W=$Arbeit$=[Ws]$
			\item $\delta Q,\Delta Q=$Wärme$=[J]$
			\item $m,m_{1},m_{2}=[kg]$
			\item $c,c_{1},c_{2}=$ spez. Wärmekapazität $=[\frac{J}{kg\cdot K}]$
			\item $C=$ Wärmekapazität $=[\frac{J}{kg}]$
			\item $\Delta T,T_{1},T_{2},T_{M}=[K]$
			\item $c_{Wasser}=4182[\frac{J}{kg\cdot K}]$
			\item $c_{Eis}=2060[\frac{J}{kg\cdot K}]$
			\item $Q_{s},Q_{v}=[J]$
			\item $q_{s},q_{v}=[\frac{J}{kg}]$
			\item $q_{s,Eis}=333700[\frac{J}{kg}]$
			\item $q_{v,Wasser}=2257000[\frac{J}{kg}]$
		\end{itemize}
		%\end{minipage}
		\\ \hline
	\end{tabular}
%\end{center}
%\end{table}

	\textbf{Gründe für eine höhere Wärmekapazität:}\\
	1.) Anzahl Freiheitsgrade\\
	2.) Grössere Anzahl an Teilchen (kleinere Dichte)\\
	\textbf{Anomalie des Wassers:}\\
	Höchste Dichte bei 4 \textcelsius $\Rightarrow$ Volumenzunahme bei Erhöhung und Verminderung der Temperatur\\
	

\subsection{Wärmetransport}		%Wärmetransport
%\begin{table}[h!]
%	\begin{center}
	\begin{tabular}{ | m{10cm} | m{8cm}  | }
		\hline
		Formeln & Einheiten \\ \hline
		\hline
		%\begin{minipage}[t]{5cm}
		\begin{itemize}
			\item $k=\dfrac{1}{\dfrac{1}{\alpha_{i}}+\sum_{s}\dfrac{d_{s}}{\lambda_{s}}+{\dfrac{1}{\alpha_{a}}}}$
			\item $k_{zylindrisch }=\dfrac{1}{r_{a}}\cdot \dfrac{1}{\dfrac{1}{r_{i}\cdot \alpha_{i}}+\sum_{s}\dfrac{1}{\lambda_{s}}\cdot ln(\dfrac{r_{sa}}{r_{si}}+\dfrac{1}{r_{a}\cdot \alpha_{a}})}$ 
			\item $P_{H}=$ Heizleistung $=k\cdot A\cdot \Delta T$
			\item $P_{K}=$ Kühlleistung $=k\cdot A+\underbrace{c_{L}\cdot \rho _{L}\cdot \dfrac{V}{t}}_{Lueftung}\cdot \Delta T$
			\item {\color{red} Auch hier gilt die Wärmebilanz $Q_{ab}=Q_{zu}$. Hier können die Aufgaben meist über den Vergleich der Heizleistungen gelöst werden.}
		\end{itemize}
		%\end{minipage}
		&
		%\begin{minipage}{5cm}
		\begin{itemize}
			\item $\alpha_{i},\alpha_{a}=$ Wärmeübergangszahl $=[\dfrac{W}{m^{2}\cdot K}]$
			\item $k=$ Wärmedurchgangszahl $=[\dfrac{W}{m_{^{2}\cdot K}}]$
			\item $d_{s}=$ Wanddicke $=[m]$
			\item $\lambda_{s}=$ Anzahl Wandschichten $=[1]$
			\item $r=$ Wandradien $=[m]$
			\item $P_{H},P_{K}=[W]$
			\item $A=$ Wandfläche $=[m]$
			\item $c_{L}=1005[\dfrac{J}{Kg\cdot K}]$
			\item $\rho_{L}=1.2041[\dfrac{kg}{m^{3}}]$
			\item $V=$ Raumvolumen $=[m^{3}]$
			\item $t=[s]$
			\item $\Delta T=[K]$
		\end{itemize}
		%\end{minipage}
		\\ \hline
	\end{tabular}
%\end{center}
%\end{table}

\subsection{Luftfeuchtigkeit}		%Luftfeuchtigkeit
%\begin{table}[h!]
	\begin{center}
		\begin{tabular}{ | m{6cm} | m{12cm}  | }
			\hline
			Formeln & Einheiten \\ \hline
			\hline
			%\begin{minipage}[t]{5cm}
			\begin{itemize}
				\item \textbf{Absolute Luftfeuchtigkeit:}
				\item $f=\dfrac{m_W}{V}$
				
				\item \textbf{Relative Luftfeuchtigkeit:}
				\item $f_R=\dfrac{f}{f_{max}}=\dfrac{m_W}{m_S}$
			
			\end{itemize}
			%\end{minipage}
			&
			%\begin{minipage}{5cm}
			\begin{itemize}
				\item $f=$ absolute Luftfeuchtigkeit $=[\frac{kg}{m^3}]$
				\item $f_R=$ relative Luftfeuchtigkeit $=[1]$
				\item $m_W=$ Wasserdampfmasse $=[kg]$
				\item $m_S=$ Wasserdampfmasse bei Sättigung $=[kg]$
				\item $V=[m^3]$
				\item $f_{max}=$ maximale Luftfeuchtigkeit $=[\frac{kg}{m^3}]$
				\item $\dfrac{\tau}{\Delta}$
			
			\end{itemize}
			%\end{minipage}
			\\ \hline
		\end{tabular}
	\end{center}
%\end{table}
\newpage

\subsection{Freiheitsgrade}				%Freiheitsgrade
%\begin{table}[h!]
	\begin{tabular}{|l|m{3cm}|m{3cm}|m{3cm}|m{3cm}|}
		\hline 
	\textbf{Atommodel}	& \textbf{Translation} & \textbf{Rotation}  & \textbf{Oszillation} & \textbf{Gesamt}    \\ 
		\hline 
		Massenpunkt & 3 & 0 & 0 & 3   \\ 
		\hline 
		Starre Hantel& 3 & 2 & 0 & 5   \\ 
		\hline 
		Schwingende Hantel& 3 & 2 & $1\cdot 2$ & 7   \\ 
		\hline 
		Mehratomig starr& 3 & 3 & 0 & 6    \\ 
		\hline 
		Kristall& 0 & 0 & $3\cdot 2$ & 6    \\ 
		\hline 
	\end{tabular} 
%\end{table}




